
\chapter{Pendolo a torsione}

6 Aprile 2011
\section{Introduzione}


\subsection{Oggetto della ricerca}
L'esperienza si prefissa l'obiettivo di misura le costanti di torsione $c$ di tre fili di sezione differente. 


\subsection{Metodo}
L'esperienza si compone di tre differenti fasi.
\begin{itemize}
\item Misura sperimentale dello spostamento angolare a seguito del momento della forza peso, al fine di calcolarne il momento d'inerzia 
\item Misura sperimentale del periodo di un pendolo di torsione, al fine di calcolare la costante di torsione dei fili
\item Misura della costante $c$ di torsione tramite l'instaurazione di un equilibrio tra il momento elastico e il momento della forza peso, e confronto con il valore teorico di $c$
\end{itemize}

\subsection{ Strumentazione e dati geometrici}

Nell'esperimento verrà utilizzato un pendolo di torsione strutturato nel seguente modo:


Micrometro: $ \pm 1 \mu m$
Metro: $\pm 1 mm$
Sensore di rotazione: $\pm 0.09$ gradi

\begin{tabular}{ll}
Masse puntiformi: & 0.074 kg\\
Lunghezza sbarra: & 0.38 m\\
Massa Sbarra: & 0.2563 kg\\
Raggio Sbarra: & 0.0045 m\\

\midrule

Massa anello: & 0.46927 kg\\
Raggio interno: & 0.0265 m\\
Raggio esterno: & 0.0355 m\\

\midrule

Supporto & 0.0035 m\\

\midrule

Massa disco: & 0.12055 kg\\
Raggio disco: & 0.0475 m\\
Diametro carrucola: & 29 mm\\

\end{tabular}


\section{Raccolta dei dati}

In primis sono stati raccolti i dati relativi alle caratteristiche geometriche dei tre fili in esame:

\begin{tabular}{l|l|l|l}

 & Filo A & Filo B & Filo C \\
\midrule
Diametro (mm) & 1.750 & 1.175 & 0.880 \\

Lunghezza (cm) & 41.3 &  43 & 33.5 \\
\midrule
\end{tabular}

\subsection{Misura dei momenti d'inerzia}

In questa fase si provvederà a ricavare  sperimentalmente i momenti d'inerzia dei seguenti corpi:
\begin{itemize}
\item Disco 
\item Anello
\item Sbarra cilindrica omogenea, con due masse uguali, scorrevoli su di essa, poste equidistanti dall’asse di rotazione
\end{itemize}

Si è utilizzato un sistema di due pulleggie e un sensore di rotazione.
Il sensore di rotazione fornisce la posizione angolare in funzione del tempo. 



- Grafico angolo/tempo cadute
- Grafico accelerazione/tempo cadute
- Calcolo momenti da dati sperimentali:





\subsection{Pendolo di torsione}

Da inserire:
- Grafico (isto+gaussiana) dei Periodi
- Grafico spazio/tempo dei Periodi


Abbiamo ricavato i seguenti periodi per ogni filo:
\subsubsection{Filo A}

\begin{verbatim}
= Period (in s) =
Mean = 0.644

String preformatted for GNU Octave:
values = [ 0.65 0.65 0.65 0.625 0.65 0.675 0.65 0.625
0.65 0.65 0.65 0.625 0.65 0.65 0.625 0.65
0.65 0.65 0.625 0.675 0.625 0.65 0.625 0.65 0.65 0.65
0.65 0.625 0.65 0.675 0.625 0.65 0.65 0.625
0.625 0.675 0.65 0.65 0.625 0.65 0.625 0.65 0.625 0.65
0.65 0.625 0.65 0.65 0.65 0.625]


\end{verbatim}


Media = 0.644

\subsubsection{Filo B}
Periodo di oscillazione (s):

\begin{center}
\begin{tabular}{llllllll}
2.4   & 2.4   & 2.425 & 2.4  & 2.4   & 2.425 & 2.4  & 2.4 \\
2.425 & 2.425 & 2.425 & 2.45 & 2.45  & 2.5   & 2.5  & 2.45 \\
\end{tabular}
\end{center}
Media = 2.43

\subsubsection{Filo C}

\begin{verbatim}
= Period (in s) =
Mean = 3.844

String preformatted for GNU Octave:
values = [ 3.875 3.9 3.9 3.875 3.9 3.875 3.875
3.875 3.875 3.85 3.875 3.85 3.85 3.85 3.825
3.825 3.825 3.8 3.8 3.75 3.775 3.75]
\end{verbatim}

\subsection{ Misura di equilibrio}


\subsubsection{Filo A,B,C}
\begin{tabular}{l|l|l|l}

\multicolumn{2}{|c|}{ciao}\\

\midrule
& A & B & C \\
50g & 4.0 & 12.0 & 31.0 \\
100g & 7.0 & 23.0 & 61.0 \\
200g & 13.0 & 47.0 & 126.0 \\
\midrule


\end{tabular}

\section{Analisi dei dati}

\subsection{Misura dei momenti d'inerzia}

Acquisiti i dati della posizione angolare in funzione del tempo, è possibile calcolare l'accelerazione angolare del sistema. A questo punto, si può calcolare $I$ nel seguente modo:
$$ bF_p = I\alpha $$
La forza peso ($F_p$) è nota, così come il braccio ($b$) e, dall'elaborazione dei dati sperimentali, possiamo calcolare $\alpha$.

seguenti corpi, elencati con i rispettivi momenti teorici:
$$ \frac{1}{2} m R^2 $$
$$ \frac{1}{2} m (R^2_1 + R^2_2) $$
$$ \frac{1}{12} m L^2 + 2 \mu D^2  $$

Grafici


Questi dati vanno confrontati con i momenti d'inerzia calcolati partendo dai dati geometrici dei corpi


$$ I_sbarra = 0.00844 $$
$$ I_disco = 0.00013 $$
$$ I_anello = 0.00046 $$

\subsection{Pendolo di torsione}


Questo ci permetterà di calcolare nella seconda parte la costante di torsione $c$ sfruttando la seguente eguaglianza:

$$ c = 4\pi^2\frac{I}{T^2} $$


$$ T = 2\pi \sqrt{\frac{I}{c}}$$




\subsection{Misura di equilibrio}

$$ c = G \frac{\pi}{2}\frac{r^4}{l} $$

dove $r$ e $l$ sono rispettivamente il raggio e la lunghezza del filo, $I$ è il momento d'inerzia, $T$ il periodo, e $G$ è il modulo di rigidità o di scorrimento ed è una proprietà specifica del materiale di cui il filo è realizzato.

Ponendo
$$ M_{peso} = bmg = c\theta $$

dove $b$ è il braccio di applicazione della forza peso, ovvero il raggio della carrucola.
I valori di c calcolati dono dunque stati (in $N\cdot m/rad$):

\begin{center}
\begin{tabular}{l|lll}
Peso & Filo A & Filo B & Filo C \\
\midrule
50g & 0.1019 & 0.0340 & 0.0135 \\
100g & 0.1164 & 0.0354 & 0.0134 \\
200g & 0.1254 & 0.0347 & 0.0129 \\
\midrule
Media & & & \\
\end{tabular}
\end{center}

Non siamo riusciti a trovare i valori di G tabulati in quanto non conoscevamo il materiale in cui era costruito, quindi abbiamo cercato di arrivare a una migliore stima per riconoscere il materiale del filo. Abbiamo utilizzato un valore medio di $c$ per calcolare $G$ secondo la seguente formula:

$$ G = \frac{2cl}{\pi r^4} $$

I valori calcolati sono stati:
\begin{center}
\begin{tabular}{lll}
Filo A & Filo B & Filo C \\
\midrule
49 GPa & 78 GPa & 73 GPa \\
\end{tabular}
\end{center}

Da cui abbiamo dedotto che il filo A fosse in titanio, e i fili B e C in acciaio.
I dati tabulati sono i seguenti
\begin{center}
\begin{tabular}{ll}
Acciaio & Titanio \\
\midrule
41 GPa & 78 GPa \\
\end{tabular}
\end{center}

\section{Conclusioni}