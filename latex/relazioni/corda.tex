\chapter{Corda Vibrante}

\section{Introduzione}

Obiettivo di questo esperimento è lo studio della propagazione di onde elastiche su una corda fissata ad un estremo, sotto l'azione di una forzante con andamento sinusoidale. Verifichiamo la dipendenza della frequenza dell'onda ($\nu_n$) dalla lunghezza della corda, dalla sua massa lineare ($\mu$) e dalla tensione ($T$) applicata su di essa. Tali grandezze sono infatti legate dalle seguenti relazioni:
\begin{center}
\begin{tabular}{c c c}
$
v= \sqrt{\frac{T}{\mu}} 
$
&
\hspace{1cm}
$
v= \nu_{n} \lambda_n
$ 
&
\hspace{1cm}
$
\lambda_n= \frac{2L}{n} 
$
\\
\end{tabular}
\end{center}

da cui:

\begin{equation}
\nu_n=\frac{n}{2L}\sqrt{\frac{T}{\mu}}
\end{equation}

con $n$ numero intero, al cui variare corrispondono differenti armoniche.


\section{Strumenti}

La strumentazione di cui ci siamo serviti si componeva di due corde:\\

\begin{tabular}{c c c c}
\textbf{Corda A} & \hspace{1cm} $L_A= 2.46\ m$ & \hspace{1cm} $m_A=5.21\cdot10^{-3}\ kg$ & \hspace{1cm} $\mu_A= 2.12\cdot10^{-3}\ kg/m$\\
\textbf{Corda B} & \hspace{1cm} $L_B= 2.50\ m$ & \hspace{1cm} $m_B=1.63\cdot10^{-2}\ kg$ & \hspace{1cm} $\mu_B= 6.52\cdot10^{-3}\ kg/m$\\
\end{tabular} 
\\

La forzante è data da un oscillatore elettromeccanico, mentre la tensione è esercitata da pesetti appesi all'estremità libera della corda, sospesi attraverso una carrucola (come illustrato in \textbf{figura}). 
L'errore associato alla misura di $L_A$ ed $L_B$ è di $\pm0.001\ m$; quello sulle masse è del tutto trascurabile. 

\section{Ricerca delle armoniche}

Nella prima parte dell'esperienza abbiamo verificato che le armoniche superiori sono date da multipli interi della frequenza fondamentale. A tal fine, dopo aver trovato la frequenza di risonanza \textbf{corrispondente ad $n=1$ (ossia la fondamentale)}, facendo variare la frequenza dell'oscillatore elettromeccanico e cercando il punto in cui la corda presentava i nodi alle due estremità e un grande ventre al centro, abbiamo misurato le frequenze relative alle prime cinque armoniche, cui corrispondono le configurazioni descritte in \textbf{figura}. 
\\

Abbiamo ripetuto la misurazione sottoponendo la corda a quattro differenti forze di tensione, e successivamente abbiamo ripetuto l'esperienza per l'altra corda, di massa lineare differente. \\
In tabella sono riportati i valori delle frequenze di risonanza registrate dall'oscillatore elettromeccanico: in verticale si può leggere la dipendenza dalla variazione di tensione per valori crescenti delle masse appese, in orizzontale dal numero di armonica.
\\

\begin{center}
\begin{tabular}{c     c}
\textbf{Corda A} & \textbf{Corda B}\\
\\
\begin{tabular}{ c | c | c | c | c | c }
Massa($g$) & 1 & 2 & 3 & 4 & 5\\
\midrule
550 & 23.7 & 45.5 & 71.3 & 94.6 & 119.6\\
450 & 21,4 & 42,6 & 63,4 & 85,3 & 108,9\\
350 & 19.6 & 37.8 & 57.4 & 76.3 & 95.0\\
250 & 16.1 & 33.1 & 48.7 & 63.4 & 81.3 \\
\end{tabular}
&
\hspace{1cm}

\begin{tabular}{ c | c | c | c | c | c }
Massa($g$) & 1 & 2 & 3 & 4 & 5\\
\midrule
1050 & 19.3 & 38.4 & 56.1 & 75.2 & 94.7 \\
550 & 14.1 & 26.8 & 40.8 & 55.0 & 69.9 \\
450 & 13.1 & 25.4 & 38.9 & 51.7 & 64.2\\
350 & 10.9 & 22.8 & 35.2 & 46.9 & 58.3\\
\end{tabular}

\end{tabular}
\end{center}


Nel seguente grafico mostriamo la relazione tra la frequenza $\nu$ e il numero di armonica $n$. Per rendere più leggibile il grafico ci siamo limitati a rappresentare tale relazione per la corda A sottoposta ad una tensione $T=(0.250\ kg)(9.81\ m/s^2)$. Come ci si aspettava si tratta di una relazione lineare, in accordo con la \textbf{1.1}

%Grafico con i punti.

Abbiamo dunque interpolato i dati con la \textbf{1.1}, utilizzando il metodo dei minimi quadrati, nel caso di una retta passante per l'origine.

\begin{center}
\begin{tabular}{c c}
$m=\frac{\displaystyle\sum{x_i y_i}}{\displaystyle\sum{x_i^2}}$ & \hspace{2cm} $\sigma_m=\frac{\displaystyle\sigma_y}{\sqrt{\displaystyle\sum{x_i^2}}}$
\end{tabular}
\end{center}

Teniamo come variabili indipendenti le $n_i$, le $y_i$ sono ovviamente le frequenze corrispondenti (l'incertezza associata è stata ricavata sperimentalmente e vale $\pm0.5\ Hz$), mentre i parametri da interpolare sono $m=\frac{1}{2l}\sqrt{\frac{T}{\mu}}$ (coefficiente angolare della retta).
$l=1.11\pm0.01\ m$ \footnote{In questo caso l'incertezza sulla lunghezza è stata stimata di un centimetro nonostante quella dello strumento fosse solo di 0.001 m, imputando tale incertezza ad eventuali errori degli sperimentatori: la misurazione infatti richiedeva di prendere come estremo il punto di tangenza con la carrucola, un'operazione in cui era difficile essere veramente accurati.}
è la distanza tra i punti di sospensione della corda: il primo coincide con il punto di applicazione della forzante, il secondo è dato dal punto di tangenza con la carrucola.  

Dall'interpolazione ricaviamo: $m=16.1\pm0.1$ Dunque, moltiplicando $m$ per $2l=2.22\pm0.02\ m$ e propagando l'errore:
$$\sigma_v=\sqrt{\left(\frac{\sigma_m}{m}\right)^2+\left(\frac{\sigma_l}{l}\right)^2}=0.35$$
otteniamo $$v=35.7\pm0.35\ m/s$$ 
Calcoliamo il  $\chi^2$ e otteniamo un valore di $0.8$. La funzione è perfettamente in accordo, in quando il $chi^2 << N$ numero di misure. 
Confrontiamo il valore trovato con quello ottenuto dall'equazione:
$$v=\sqrt{\frac{T}{\mu}}=\sqrt{\frac{2.45\ N}{2.12 \cdot10^{-3}\ kg/m}}=34.0\ m/s$$

L'errore è:
$$4\% \leq \frac{\delta_v}{v} \leq 5\% $$




\begin{center}

\includegraphics[scale=0.5]{../grafici/corda_1armonica}
\end{center}


\section{Dipendenza dalla tensione}
Dall'equazione $1.1$ si deduce la dipendenza quadratica tra la tensione $T$ e la frequenza $\nu$. 
Verifichiamo questa dipendenza utilizzando i dati della seconda armonica misurati precedentemente. 
\begin{center}
\begin{tabular}{c | c}

\textbf{Corda A}
\begin{tabular}{c|c}
Tensione $N$ & Frequenza $Hz$ \\
\midrule
2.45 & 33.1\\
3.43 & 37.8\\
4.41 &42.6\\
5.39 &45.6\\
\end{tabular}
&\textbf{Corda B}
\begin{tabular}{c|c}
Tensione $N$ & Frequenza $Hz$ \\
\midrule
3.45 & 22.8\\
4.41 & 25.4\\
5.39 & 26.8\\
10.3 & 38.4\\
\end{tabular}
\\
\end{tabular}
\end{center}


 Di seguito il grafico che rappresenta la relazione tra $\nu^2$ e $\tau$.
\begin{center}
\includegraphics[scale=0.5]{../grafici/corda_tensione}
\end{center}

Si può chiaramente vedere la dipendenza lineare, tra $\nu^2$ e $\tau$. 


\section{Dipendenza dalla lunghezza}
L'ultima analisi riguarda la dipendeza di $\nu_n$ dalla lunghezza della corda, per una data tensione fissata. Essa è ancora una conseguenza della $1.1$, dalla quale si deduce che le due grandezze sono legate da proporzionalità inversa.\\
 
Di seguito sono riportati in tabella i valori raccolti per la prima armonica, per le corde A e B; sono stati scelti valori differenti per le due tensioni in quanto la corda a massa lineare maggiore aveva una risposta poco significativa quando soggetta alla tensione che si era usata per la corda A.\\

\begin{center}


\begin{tabular}{c c}

\textbf{Corda A},  $T=5.40\ N$ & \hspace{3cm} \textbf{Corda B},  $T=10.3\ N$\\
\\
\begin{tabular}{|c|c|}
\toprule
L($m$) & $\nu (Hz) $ \\
\midrule
1.05 & 24.6\\
1.11 & 23.7\\
1.17 & 22.8 \\
1.23 & 20.9 \\
1.32 & 20.1 \\
\bottomrule
\end{tabular}
 
& \hspace{3cm}

\begin{tabular}{|c|c|}
\toprule
L($m$) & $\nu (Hz) $ \\
\midrule
1.10 & 19.33 \\
1.15 & 18.25 \\
1.17 & 17.75 \\
1.22 & 17.00 \\
1.34 & 15.69 \\
1.53 & 13.92 \\
1.60 & 12.96 \\
0.91 & 22.83 \\
0.75 & 27.85 \\
0.60 & 35.00 \\
0.50 & 42.67 \\
\bottomrule
\end{tabular}

\end{tabular}

\end{center}



Il grafico seguente illustra il comportamento della corda B.

\includegraphics[scale=0.75]{"../grafici/CordaPrimaArmonica"}


\section{Conclusioni}

Abbiamo verificato l'equazione 1.1, mostrando la dipendenza della frequenza dalla lunghezza, dalla tensione e dalla massa lineare.





\chapter*{Appendice}
\section*{Metodo di analisi dei dati informatizzato}
Per l'analisi dei dati sperimentali si è fatto largo uso di strumenti informatici ausiliari.
In particolare, è stata usata l'applicazione DataStudio per la raccolta di dati da sensori elettronici, ed occasionalmente per l'interpolazione dei dati.
Per quanto riguarda l'analisi degli errori, si è preferita una soluzione costruita tramite il pacchetto software open source "Sage", distribuito sotto i termini della GNU General Public License.

Questo strumento utilizza il linguaggio di programmazione Python, e alcune librerie esterne. Di seguito viene allegato e commentato il programma principale, scritto da noi.

\begin{lstlisting}
reset()

format = "png"
dpis = 90

import numpy as np
import matplotlib.pyplot as plt
\end{lstlisting}

Questa sezione è un preambolo: vengono importate le librerie principali, viene impostato il formato di output desiderato (in questo caso "png", comodo per lo sviluppo) e la risoluzione di questo output.

\begin{lstlisting}
class Analyzer:
    uuid = "" # internally used
    caption = "" # caption for the histogram/plot
    color = '#50e300' # histogram color
    nOfBins = 5; # number of bins (default: 5)
    data = [] # array with data. if you're using integer values,
    		  # make sure at least one is a float.
    x = "" # x-axis label, latex syntax
    y = "" # y-axis label, latex syntax
    
    plotText = "" # optional: print this text on the plot - latex syntax
    xUnitOfMeasure = "" # optional: unit of measure of x
    
\end{lstlisting}

Questa è la classe base: si occupa di disegnare il grafico e di fornire l'analisi statistica di base.

\begin{lstlisting}
    def setTitle(self, caption):
        self.caption = caption
        self.uuid = caption.replace('\\', '').replace(' ', '')
        
    # Note: make sure xUnitOfMeasure is set before calling this
    def quickPrepare(self, list):
        m = self.xUnitOfMeasure
        self.y = "f(k)/\\Delta k"
        self.data = list['data']
        self.setTitle(d['uuid'])
        if('bins' in d):
            self.nOfBins = d['bins']
        self.plotText = '\mu = %.5f %s;\ \sigma = %.5f %s;'\
                        % (mean(self.data), m, std(self.data), m)
        
    def plotGauss(self):
        a = self.data
        
        plt.clf()
        mu = mean(a)
        sigma = std(a)
        n, bins, patches = plt.hist(a, self.nOfBins, normed=1, facecolor=self.color, alpha=0.75)
        
        x = np.arange(min(bins)-10*sigma, max(bins)+10*sigma, 0.02*(max(bins)-min(bins)));
        y = 1/(sigma*np.sqrt(2*np.pi)) * np.exp( -((x-mu)^2 / (2* (sigma^2) )) );
        l = plt.plot(x, y, 'k--', linewidth=1)
        
        plt.xlabel(r"$"+self.x+"$")
        plt.ylabel(r"$"+self.y+"$")
        
        plt.title(r"$"+self.caption.replace(' ', '\\ ')+"$")
        plt.axis([mu-3*sigma, mu+3*sigma, 0, 3/(sigma*np.sqrt(2*np.pi))])
        
        plt.annotate(r"$"+self.plotText+"$", xy=(0.075,0.85), xycoords='axes fraction')
        
        plt.grid(True)
        plt.savefig(self.uuid+"."+format, dpi=dpis)
        
    def statAnalysis(self):
        stderr = "%.5f" % (std(self.data)/np.sqrt(len(self.data)))
        media = "%.5f" % mean(self.data)
        stddev = "%.5f" % std(self.data)
        print "=== "+self.caption+" ==="
        print "Media: "+ media +" - Std. Dev.: "+ stddev
        print "Errore della media: "+ stderr
\end{lstlisting}