\chapter{Corda Vibrante}

\section{Introduzione}

Obiettivo di questo esperimento è lo studio della propagazione di onde elastiche su una corda \textbf{fissata ad un estremo}, sotto l'azione di una forzante con andamento sinusoidale. \textbf{Verifichiamo} la dipendenza della frequenza dell'onda ($\nu_n$) \textbf{dalla lunghezza della corda, dalla sua massa lineare ($\mu$)} e dalla tensione ($T$) applicata su di essa.
\textbf{Tali grandezze sono infatti legate dalle seguenti relazioni:}
\begin{center}
\begin{tabular}{c c c}
$
v= \sqrt{\frac{T}{\mu}} 
$
&
\hspace{1cm}
$
v= \nu_{n} \lambda_n
$ 
&
\hspace{1cm}
$
\lambda_n= \frac{2L}{n} 
$
\\
\end{tabular}
\end{center}

da cui:

\begin{equation}
\nu_n=\frac{n}{2L}\sqrt{\frac{T}{\mu}}
\end{equation}

con $n$ numero intero, al cui variare corrispondono differenti armoniche.


\section{Strumenti}

La strumentazione di cui ci siamo serviti si componeva di due corde:\\

\begin{tabular}{c c c}
$L_A= 2.46 m$ & \hspace{1cm} $m_A=5.21\cdot10^{-3} kg$ & \hspace{1cm} $\mu_A= 2.12\cdot10^{-3} kg/m$\\
$L_B= 2.50 m$ & \hspace{1cm} $m_B=1.63\cdot10^{-2} kg$ & \hspace{1cm} $\mu_B= 6.52\cdot10^{-3} kg/m$\\
\end{tabular} 
\\

La forzante è data da un oscillatore elettromeccanico, mentre la tensione è esercitata da pesetti appesi all'estremità libera della corda, sospesi attraverso una carrucola (come illustrato in \textbf{figura}). 
 

\section{Ricerca delle armoniche}

Nella prima parte dell'esperienza abbiamo verificato che le armoniche superiori sono date da multipli interi della frequenza fondamentale. A tal fine, dopo aver trovato la frequenza di risonanza \textbf{corrispondente ad $n=1$ (ossia la fondamentale)}, facendo variare la frequenza dell'oscillatore elettromeccanico e cercando il punto in cui la corda presentava i nodi alle due estremità e un grande ventre al centro, abbiamo misurato le frequenze relative alle prime cinque armoniche, cui corrispondono le configurazioni descritte in \textbf{figura}. 
\\

Abbiamo ripetuto la misurazione sottoponendo la corda a quattro differenti forze di tensione, e successivamente abbiamo ripetuto l'esperienza per l'altra corda, di massa lineare differente. \\
In tabella sono riportati i valori delle frequenze di risonanza registrate dall'oscillatore elettromeccanico: in verticale si può leggere la dipendenza dalla variazione di tensione per valori crescenti delle masse appese, in orizzontale dal numero di armonica.
\\

\begin{center}
\begin{tabular}{c     c}
\textbf{Corda A}
\begin{tabular}{ c | c | c | c | c | c }
Massa($g$) & 1 & 2 & 3 & 4 & 5\\
\midrule
550 & 23.7 & 45.5 & 71.3 & 94.6 & 119.6\\
450 & 21,4 & 42,6 & 63,4 & 85,3 & 108,9\\
350 & 19.6 & 37.8 & 57.4 & 76.3 & 95.0\\
250 & 16.1 & 33.1 & 48.7 & 63.4 & 81.3 \\
\end{tabular}
&
\hspace{2cm}
\textbf{Corda B}
\begin{tabular}{ c | c | c | c | c | c }
Massa($g$) & 1 & 2 & 3 & 4 & 5\\
\midrule
1050 & 19.3 & 38.4 & 56.1 & 75.2 & 94.7 \\
550 & 14.1 & 26.8 & 40.8 & 55.0 & 69.9 \\
450 & 13.1 & 25.4 & 38.9 & 51.7 & 64.2\\
350 & 10.9 & 22.8 & 35.2 & 46.9 & 58.3\\
\end{tabular}

\end{tabular}
\end{center}
\section{Dipendenza dalla tensione}
Dall'equazione $1.1$ si deduce la dipendenza quadratica tra la tensione $\tau$ e la frequenza $\nu$. 
Verifichiamo questa dipendenza utilizzando i dati della seconda armonica misurati precedentemente. 
\begin{center}
\begin{tabular}{c | c}

\textbf{Corda A}
\begin{tabular}{c|c}
Tensione $N$ & Frequenza $Hz$ \\
\midrule
2.45 & 33.1\\
3.43 & 37.8\\
4.41 &42.6\\
5.39 &45.6\\
\end{tabular}
&\textbf{
Corda B}
\begin{tabular}{c|c}
Tensione $N$ & Frequenza $Hz$ \\
\midrule
3.45 & 22.8\\
4.41 & 25.4\\
5.39 & 26.8\\
10.3 & 38.4\\
\end{tabular}
\\
\end{tabular}
\end{center}

Nel seguente grafico mostriamo la relazione tra la frequenza $\nu$ e il numero di armonica $n$. Per rendere più leggibile il grafico ci siamo limitati a rappresentare tale relazione per la corda A sottoposta ad una tensione $T=(0.250\ kg)(9.81\ m/s^2)$. Come ci si aspettava si tratta di una relazione lineare, in accordo con la \textbf{1.1}

%Grafico con i punti.

Abbiamo dunque interpolato i dati con la \textbf{1.1}, utilizzando il metodo dei minimi quadrati, nel caso di una retta passante per l'origine.

\begin{center}
\begin{tabular}{c c}
$m=\frac{\displaystyle\sum{x_i y_i}}{\displaystyle\sum{x_i^2}}$ & \hspace{2cm} $\sigma_m=\frac{\displaystyle\sigma_y}{\sqrt{\displaystyle\sum{x_i^2}}}$
\end{tabular}
\end{center}

Teniamo come variabili indipendenti le $n_i$, le $y_i$ sono ovviamente le frequenze corrispondenti (l'incertezza associata è stata ricavata sperimentalmente e vale $\pm0.5\ Hz$), mentre i parametri da interpolare sono $m=\frac{1}{2l}\sqrt{\frac{T}{\mu}}$ (coefficiente angolare della retta).
$l=1.11\pm0.01\ m$ \footnote{In questo caso l'incertezza sulla lunghezza è stata stimata di un centimetro nonostante quella dello strumento fosse solo di 0.001 m, imputando tale incertezza ad eventuali errori degli sperimentatori: la misurazione infatti richiedeva di prendere come estremo il punto di tangenza con la carrucola, un'operazione in cui era difficile essere veramente accurati.}
è la distanza tra i punti di sospensione della corda: il primo coincide con il punto di applicazione della forzante, il secondo è dato dal punto di tangenza con la carrucola.  

Dall'interpolazione ricaviamo: $m=16.1\pm0.07$ Dunque, moltiplicando $m$ per $2l=2.22\pm0.02\ m$ e propagando l'errore:
$$\sigma_v=\sqrt{\left(\frac{\sigma_m}{m}\right)^2+\left(\frac{\sigma_l}{l}\right)^2}=0.35$$
otteniamo $$v=35.7\pm0.35\ m/s$$ 
Confrontiamo il valore trovato con quello ottenuto dall'equazione:
$$v=\sqrt{\frac{T}{\mu}}=\sqrt{\frac{2.45\ N}{2.12 \cdot10^{-3}\ kg/m}}=34.0\ m/s$$

L'errore è:
$$4\% \leq \frac{\delta_v}{v} \leq 5\% $$






\begin{center}

\includegraphics[scale=0.5]{../grafici/corda_1armonica}
\end{center}




Essendo il numero dei dati troppo ridotto per essere interpolato con successo dalla curva corrispondente ($1.1$), consideriamo un'interpolazione lineare di tipo $y = mx$ dove   $y= \nu^2$ e $x = \tau$.
\\
Utilizziamo il metodo dei minimi quadrati, ottenendo per la\textbf{ corda A} $m_A = 403.95$ con incertezza $\sigma_m =0.03$ e per la \textbf{corda B} $m_B=143.0$ con incertezza $\sigma_m$.  Di seguito il grafico che rappresenta la relazione tra $\nu^2$ e $\tau$.

\begin{center}
\includegraphics[scale=0.5]{../grafici/corda_tensione}
\end{center}



\section{Dipendenza dalla lunghezza}
Sempre dall'equazione $1.1$, si deduce la proporzionalità inversa tra la frequenza $\nu$ e la lunghezza $l$. 
Corda A,550g

\begin{center}


\begin{tabular}{c|c}
L($cm$) & $\nu (Hz) $ \\
\midrule
105 & 24.6\\
111 & 23.7\\
116.5 & 22.8 \\
122.5 & 20.9 \\
131.5 & 20.1 \\
\end{tabular}
\end{center}

In tabella sono riportati i dati della frequenza delle prima armonica, 
Corda B,1050g

\begin{center}
\begin{tabular}{|c|c|}
\toprule
Lunghezza (m) & Frequenza (Hz) \\
\midrule
1.10 & 19.33 \\
1.15 & 18.25 \\
1.17 & 17.75 \\
1.22 & 17.00 \\
1.34 & 15.69 \\
1.53 & 13.92 \\
1.60 & 12.96 \\
0.91 & 22.83 \\
0.75 & 27.85 \\
0.60 & 35.00 \\
0.50 & 42.67 \\
\bottomrule
\end{tabular}
\end{center}

Inserendo questi dati su di un grafico, e interpolando la curva con Sage, troviamo il seguente grafico.

\includegraphics[scale=0.75]{"../grafici/CordaPrimaArmonica"}

Nota: $\rho$ è lasciato parametro libero, mentre invece $\tau$ è pari a 10.3 N, dato da un peso di 1.050 Kg sospeso a un'estremità della corda.

\section{Dipendenza tra frequenza e massa lineare}

