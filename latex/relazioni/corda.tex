

\chapter{Corda Vibrante}

\section{Introduzione}

\section{Ricerca delle armoniche}

\subsection{Raccolta dati}

Nella prima parte dell'esperienza, dopo aver trovato la frequenza di risonanza, abbiamo misurato le frequenze relative alle prime cinque armoniche di una corda vibrante di lunghezza 111 cm e peso 5.21 g. Abbiamo ripetuto la misurazione sottoponendo la corda a quattro differenti forze di tensione (ottenute appendendo un peso ad una delle estremità della corda, come illustrato in figura)
[ref figura]\\

\begin{center}

\begin{tabular}{ c | c | c | c | c | c }
Massa($g$) & 1 & 2 & 3 & 4 & 5\\
\midrule
550 & 23.7 & 45.5 & 71.3 & 94.6 & 119.6\\
450 & 21,4 & 42,6 & 63,4 & 85,3 & 108,9\\
350 & 19.6 & 37.8 & 57.4 & 76.3 & 95.0\\
250 & 16.1 & 33.1 & 48.7 & 63.4 & 81.3 \\
\end{tabular}
\\

\begin{tabular}{ c | c | c | c | c | c }
Massa($g$) & 1 & 2 & 3 & 4 & 5\\
\midrule
1050 & 19.3 & 38.4 & 56.1 & 75.2 & 94.7 \\
550 & 14.1 & 26.8 & 40.8 & 55.0 & 69.9 \\
450 & 13.1 & 25.4 & 38.9 & 51.7 & 64.2\\
350 & 10.9 & 22.8 & 35.2 & 46.9 & 58.3\\
\end{tabular}
\end{center}



Velocità di propagazione di un'onda elastica in un mezzo materiale:
\begin{center}


\begin{tabular}{c|c}
L($cm$) & $\nu (Hz) $ \\
\midrule
105 & 24.6\\
111 & 23.7\\
116.5 & 22.8 \\
122.5 & 20.9 \\
131.5 & 20.1 \\
\end{tabular}
\end{center}

\begin{equation}
v=\sqrt{\frac{\tau}{\mu}}
\end{equation}

scegliendo come $\tau$ la tensione e come $\mu$ la massa per unità di lunghezza della corda\\

Le frequenze delle armoniche di ordine superiore sono date dall'equazione:
\begin{equation}
\nu=\frac{n}{2L}\sqrt{\frac{\tau}{\pi\rho}}
\end{equation}

dove L è la distanza tra i due punti di sospensione, $n\in \mathbb{N}$ è il numero dell'armonica, $\rho$ la densità della corda e $\tau$ la tensione.

In tabella sono riportati i dati della frequenza a cui è stata trovata la prima armonica.

\begin{center}
\begin{tabular}{|c|c|}
\toprule
Lunghezza (m) & Frequenza (Hz) \\
\midrule
1.10 & 19.33 \\
1.15 & 18.25 \\
1.17 & 17.75 \\
1.22 & 17.00 \\
1.34 & 15.69 \\
1.53 & 13.92 \\
1.60 & 12.96 \\
0.91 & 22.83 \\
0.75 & 27.85 \\
0.60 & 35.00 \\
0.50 & 42.67 \\
\bottomrule
\end{tabular}
\end{center}

Inserendo questi dati su di un grafico, e interpolando la curva con Sage, troviamo il seguente grafico.

\includegraphics[scale=0.75]{"../grafici/CordaPrimaArmonica"}

Nota: $\rho$ è lasciato parametro libero, mentre invece $\tau$ è pari a 10.3 N, dato da un peso di 1.050 Kg sospeso a un'estremità della corda.
