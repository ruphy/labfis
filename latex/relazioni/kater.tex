\chapter{Pendolo di Kater}
\subsection{Scopo dell'esperimento}
Misura di g
\subsection{Metodo}
\subsection{Strumenti}
Pendolo di Kater (struttura definita in seguito)
$d(m_a,c_1)=15.4 cm$
Cronometro digitale con fotocellula a precisione $\pm 0.1 \cdot 10^{-3}\ s$ 
\section{Raccolta Dati}

\begin{center}
\begin{tabular}{*{5}{|c}|}
\toprule
\textbf{$d(c_1)=15cm$} & \textbf{$d(c_1)=17.9cm$} & \textbf{$d(c_1)=84.4cm$}& \textbf{$d(c_1)=81.3cm$}& \textbf{$d(c_1)=12cm$}\\
\midrule
 \textbf{Coltello1}&\textbf{Coltello1}&\textbf{Coltello1}&\textbf{Coltello1}&\textbf{Coltello1}\\
 1.9809 & 1.9211 & 1.9934 & 1.9735 & 2.0040\\
 1.9796 & 1.9207 & 1.9932 & 1.9744 & 2.0041\\
 1.9816 & 1.9264 & 1.9931 & 1.9746 & 2.0043\\
 1.9794 & 1.9253 & 1.9932 & 1.9755 & 2.0062\\
 1.9821 & 1.9113 & 1.9932 & 1.9765 & 2.0060\\
 1.9805 & 1.9278 & 1.9934 & 1.9769 & 2.0052\\
 1.9802 & 1.9278 & 1.9937 & 1.9770 & 2.0050\\
 1.9804 & 1.9277 & 1.9941 & 1.9760 & 2.0061\\
 1.9810 & 1.9278 & 1.9937 & 1.9754 & 2.0046\\
 1.9831 & 1.9278 & 1.9935 & 1.9750 & 2.0058\\

 \textbf{Coltello2}&\textbf{Coltello2}&\textbf{Coltello2}&\textbf{Coltello2}&\textbf{Coltello2}\\
 1.9934 & 1.9882 & 1.9946 &	1.9802 & \\
 1.9946 & 1.9902 & 1.9939 &	1.9812 & \\
 1.9911 & 1.9907 & 1.9943 &	1.9820 & \\
 1.9919 & 1.9908 & 1.9941 &	1.9818 & \\
 1.9940 & 1.9907 & 1.9940 &	1.9819 & \\
 1.9923 & 1.9907 & 1.9952 &	1.9813 & \\
 1.9939 & 1.9888 & 1.9954 &	1.9816 & \\
 1.9934 & 1.9898 & 1.9943 &	1.9818 & \\
 1.9914 & 1.9898 & 1.9947 &	1.9812 & \\
 1.9958 & 1.9878 & 1.9941 &	1.9810 & \\
\bottomrule
\end{tabular}
\end{center}

Abbiamo ricavato i seguenti valore medi e dev.standard 

\begin{center}
\begin{tabular}{c   c}
\begin{tabular}{|c|c|c|}
$d(m_b,c_1)$ & Periodo in c1 & $\sigma$ \\
\midrule
6.9 & 2.1108 &\\
11.7 & 2.0075 &\\
13.5 & &\\
15.0 &  1.9806 &\\
17.9 & 1.9244 &\\
21.0 & 1.9076 &\\
81.3 & 1.9755&\\
84.4 & 1.9935&\\
\end{tabular}
&
\begin{tabular}{|c|c|c|}
$d(m_b,c_1)$ & Periodo in c2 & $\sigma$ \\
\midrule
6.9 & 2.1108 &\\
11.7 & 2.0075 &\\
13.5 & &\\
15.0 &  1.9806 &\\
17.9 & 1.9244 &\\
21.0 & 1.9076 &\\
81.3 & 1.9755&\\
84.4 & 1.9935&\\
\end{tabular}
\end{tabular}
\end{center}
CONTROLLA I VALORI
$\pm\sigma$

\section{Caduta libera}

\subsection{Descrizione dell'esperimento}

\subsection{Raccolta dati}
Tutti i valori riportati in tabella sono in millisecondi.
\begin{center}
\begin{tabular}{r|*{14}{c}}
\textbf{20 cm} & 197 & 199 & 197 & 200 & 199 & 203 & 200 & 203 & 196 & 199 & 196 & 199 & 197 & 205\\
& 198 & 199 & 198 & 197 & 199 & 198 & 197 & 198 & 200 & 198 & 199 & 199 & 198 & 204\\
\midrule
\textbf{35 cm} & 265 & 265 & 265 & 261 & 262 & 263 & 266 & 262 & 269 & 266 & 261 & 263 & 262 & 261\\
\midrule
\textbf{50 cm} & 314 & 315 & 314 & 321 & 319 & 316 & 315 & 315 & 315 & 314 & 314 & 315\\
\midrule
\textbf{60 cm} & 347 & 345 & 346 & 347 & 344 & 344 & 348 & 345 & 346 & 344 & 345 & 345\\
\midrule
\textbf{90 cm} & 424& 424& 424& 423& 425& 424& 426& 423& 424& 423& 422& 425& 423\\
\end{tabular}
\end{center}

\begin{center}
\includegraphics[scale=0.75]{../grafici/20cm.png}
$$\sigma_{\bar{x}} = 0.430\ ms$$
$$\mathrm{Stima\ di\ g} = 10.10\ m/s^2$$
$$\mathrm{Stima\ (corretta)\ di\ g} = 9.80\ m/s^2 $$
$$\frac{\Delta g_c}{g} = 0.00072$$
\includegraphics[scale=0.75]{../grafici/90cm.png}
$$\sigma_{\bar{x}} = 0.296\ ms $$
$$\mathrm{Stima\ di\ g} = 10.02\ m/s^2$$
$$\mathrm{Stima\ (corretta)\ di\ g} = 9.88\ m/s^2 $$
$$\frac{\Delta g_c}{g} = 0.00707$$
\end{center}

La correzione applicata è di 3 ms aggiunti al tempo segnato dal cronometro. Non avendo dati più precisi sulla costruzione del cronometro stesso, assumiamo questo tempo come [delay]... scrivi meglio!

\begin{center}
\begin{tabular}{c|c|c|c|c}
$h$ (m) & $g$ (m/s$^2$) & $g_c$ (m/s$^2$) & $\Delta g/g$ & $\Delta g_c/g$\\
\midrule
0.20 & 10.10 & 9.80 & 0.02964 & 0.00072 \\
0.35 & 10.07 & 9.85 & 0.02659 & 0.00362 \\
0.50 & 10.04 & 9.85 & 0.02354 & 0.00435 \\
0.60 & 10.05 & 9.88 & 0.02475 & 0.00718 \\
0.90 & 10.02 & 9.88 & 0.02138 & 0.00707 \\
\end{tabular}
\end{center}


