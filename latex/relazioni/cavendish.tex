\chapter{Bilancia di Cavendish}
\section{Introduzione}
\subsection{Oggetto della ricerca}
Misurazione della costante di gravitazione G mediante la bilancia di Cavendish.
\subsection{Proprietà geometriche dei corpi e strumentazione}

\begin{tabular}{ll}
m=38.3$\pm$0.2 g & massa sfere piccole\\
M=1500$\pm$10 g	 & massa sfere grandi\\
r=9.53 mm & raggio sfere piccole\\
R=31.9 mm	 & raggio sfere grandi\\
$m_c\cong 2m$ & massa manubrio\\
\end{tabular}
\\
Il momento d'inerzia $I$ del corpo si calcola:

$$ I=2m(d^2+2/5r^2)$$
L'inerzia del corpo è:
$I = 1.94283 \cdot 10^{-4} kg\cdot m^2$
L'incertezza sul momento di inerzia è nell'ordine di $10^{-6}$, perciò è trascurabile. 
\subsection{Metodo in breve}



\section{Misura $k$ e $\theta$}
L'incertezza relativa allo strumento per la misurazione del periodo è trascurabile rispetto l'entità della misura. 
Il periodo misurato è:

$$T^2 = 4 \pi^2 \frac{I}{k}$$
Quindi
$$k = 4 \pi^2 \frac{I}{T^2}$$

$$k = 2.84 \cdot 10^{-8} kg \cdot \frac{m^2}{s^2}$$

Per trovare la posizione di equilibrio, $S_1$ utilizziamo la formula ricorsiva:

$$X_m = \frac{\frac{X_1+X_3}{2} +X_2}{2}$$ 

\begin{tabular}{c|*{7}{c}}
$2k\pi$ & \\
\midrule
$(2k+1)\pi$ & \\
\end{tabular}

$$G = -\frac{k \theta b^2}{2mMd} $$

\section{Analisi dati} 

\section{Conclusioni}
