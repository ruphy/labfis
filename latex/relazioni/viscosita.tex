
\chapter{Viscosità}

\subsection{Da fare}
\begin{itemize}
 \item Grafici con barrette di errore
\end{itemize}

\section{Preambolo}
\subsection{Obiettivo di ricerca}
\subsection{Strumenti di laboratorio}

\section{Misurazioni preliminari}
\subsection{Sferette}
Abbiamo raccolto le seguenti misurazioni per quanto riguarda il diametro delle sferette:
\begin{center}
\begin{tabular}{|l|l|}
\toprule
 Sferette tipo 1: & 3mm\\
 Sferette tipo 2: & 4mm\\
 Sferette tipo 3: & 5mm\\
 Sferette tipo 4: & 6mm\\
\bottomrule
\end{tabular}
\end{center}

La sensibilità del calibro è di 0.05mm.\\
Peso delle sferette:

\begin{center}
\begin{tabular}{lll}
 Sferette tipo 1:
 & 0.1102g & 0.1102g \\
 & 0.1101g & \\
 \midrule
 Sferette tipo 2: 
 & 0.2612g & 0.2613g \\
 & 0.2613g & 0.2612g \\
 \midrule
 Sferette tipo 3: 
 & 0,5098g & 0,5100g \\
 & 0,5094g & 0,5100g \\
 & 0,5096g & 0,5098g \\
 \midrule
 Sferette tipo 4:
  & (3,521g/4) & 0.8806g \\
  & 0.8802g & 0.8807g \\
\end{tabular}
\end{center}

La sensibilità della bilancia è di $10^{-6}$ kg

In particolare, per una sferetta singola da 5mm, abbiamo ottenuto le seguenti misurazioni\\
\begin{tabular}{ccccccc}
0,5096g & 0,5098g & 0.5099g & 0,5095g & 0,5095g & 0,5096g & 0,5097g
\end{tabular}
\\
Per quanto riguarda il cilindro di glicerina, abbiamo segnato due tacche a una distanza di 15,1mm l'una dall'altra.
Per misurare questa distanza abbiamo utilizzato un metro a nastro con una sensibilità di 1mm.

\subsection{Dati sperimentali}
Di seguito le tabelle che registrano il tempo impiegato per attraversare il fluido.
\subsubsection{Sferette da 6mm}
\begin{tabular}{|c|c|c|c|c|c|c|c|c|c|}
\toprule
 1,68s & 1,68s & 1,65s & 2,21s* & 1,68s & 1,68s & 1,62s & 1,71s & 1,71s & 1,75s \\
 1,78s & 1,71s & 1,71s & 1,65s & 1,59s  & 1,65s  & 1,71s  & 1,68s & 1,68s & 1,65s \\
\midrule
 1,62s & 1,56s & 1,71s & 1,71s & 1,68s & 1,62s & 1,59s & 1,59s & 1,53s & 1,59s \\
 1,46s & 1,62s & 1,59s & 1,56s & 1,62s & 1,56s & 1,62s & 1,43s & 1,53s & 1,59s \\
\midrule
 1,56s & 1,56s & 1,53s & 1,62s & 1,68s & 1,56s & 1,56s & 1,59s & 1,59s & 1,50s\\
 1,53s & 1,62s & 1,50s & 1,56s & 1,59s & 1,62s & 1,43s & 1,62s & 1,43s & 1,40s \\
\midrule
 1,59s & 1,46s & 1,46s & 1,59s & 1,65s & 1,59s & 1,68s & 1,43s & 1,53s & 1,62s \\
 1,46s & 1,50s & 1,56s & 1,59s & 1,50s & 1,59s  & 1,59s & 1,59s & 1,53s & 1,59s \\
\midrule
1,59s & 1,46s & 1,53s & 1,43s & 1,53s & 1,46s & 1,46 & 1,56 & 1,40s & 1,34s \\
1,46s & 1,37s  & 1,50s   & 1,50s  & 1,43s & 1,31s & 1,40s  & 1,43s  & 1,34s & 1,43s \\
\bottomrule
\end{tabular}
\subsubsection{Sferette da 5mm}
\begin{tabular}{|c|c|c|c|c|c|c|c|c|c|}
\toprule
 2,34s* & 2,43s* & 2,31s & 2,21s & 2,12s & 2,53s* & 2,21s & 2,21s & 2,71s* & 2,71s* \\
 2,12s & 2,34s & 2,18s & 2,34* & 2,12s & 2,06s & 2,12s & 2,59s* & 2,03s & 2,06s \\
\midrule
 2,87s* & 2,21s & 2,21s & 2,12s & 2,03s & 2,09s & 2,12s & 2,46* & 2,06s & 2,25s \\
 2,15s & 2,25s & 2,00s &  &  &  &  &  &  & \\
\bottomrule
\end{tabular}
\subsubsection{Sferette da 4mm}
\begin{tabular}{|c|c|c|c|c|c|c|c|c|c|}
\toprule
 3,06s & 3,09s & 3,15s & 3,12s & 3,03s & 3,15s & 3,25s & 3,25s & 3,18s & 3,09s \\
 3,09s & 3,09s & 3,21s & 3,12s & 3,06s & 3,06s & 3,15s & 3,46s & 3,25s & 3,12s \\
\bottomrule
\end{tabular}
\subsubsection{Sferette da 3mm}

\begin{tabular}{|c|c|c|c|c|c|c|c|c|c|}
\toprule
 5,28s & 5,09s & 4,96s & 5,25s & 5,28s & 5,12s & 5,06s & 5,12s & 5,06s & 5,28s \\
 5,03s & 5,06s & 4,87s & 4,90s & 4,96s & 5,06s & 5,06s & 4,95s & 4,93s & 5,18s \\
 5,18s & 5,18s & 5,21s & 4,87 & 5,34s &  &  &  &  & \\
\bottomrule
\end{tabular}

* = per queste misurazioni, la pallina si è avvicinata molto alla parete del tubo, e dunque le misurazioni potrebbero essere falsate.
TODO: rimuovi i vecchi valori

\section{Elaborazione dei dati}
\subsection{Calcolo dei tempi}
Per le sferette da 6mm abbiamo scartato il valore 2,21s, connotato dall'asterisco, in quanto evidente errore sperimentale. A conferma di ciò, si trova che esso esso dista circa 6,5 deviazioni standard dal valore medio.

[tabelle statistiche]
- Gaussiane+istogrammi
- Plotting dei dati
- Tabelle di conti (frequenze assolute, relative, ...)

\subsubsection{Note operative}
Abbiamo eliminato alcuni dati

\subsection{Estrapolazione}
- Grafici velocità (lin, quad e log)
- Calcolo di astar, bstar

\subsection{Sezione teorica}
Abbiamo usato le seguenti formule statistiche per....

\subsection{Conclusioni}
