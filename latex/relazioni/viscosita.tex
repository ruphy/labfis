\chapter{Viscosità}
\section{Introduzione}
\subsection{Obiettivo di ricerca}
L'obiettivo della ricerca è misurare l'indice di viscosità del fluido all'interno di un cilindro, tramite la misura dei tempi di caduta di una serie di sferette di acciaio di massa e diametro noti. 
\subsection{Strumenti di laboratorio}

Il diametro della sfere è stato misurato con un calibro, e l'incertezza associata alla misura è di $\pm0.05 mm$. Il peso è stato misurato con una bilancia, e l'incertezza associata alla misura è  $\pm10^{-6}$ kg

\begin{center}
\begin{tabular}{lll}
 \textbf{Sferette da 3mm:}
 & 0.1102g & 0.1102g \\
 & 0.1101g & \\
 \midrule
\textbf{Sferette da 4mm: }
 & 0.2612g & 0.2613g \\
 & 0.2613g & 0.2612g \\
 \midrule
\textbf{Sferette da 5mm: }
 & 0,5098g & 0,5100g \\
 & 0,5094g & 0,5100g \\
 & 0,5096g & 0,5098g \\
 \midrule
 \textbf{Sferette da 6mm:}
  & (3.521g/4) & 0.8806g \\
  & 0.8802g & 0.8807g \\
\end{tabular}
\end{center}
Ai fini dell'esperimento, le incertezze associate alle misure della massa e del diametro sono sufficientemente piccole da risultare trascurabili. Nell'analisi dei dati verranno quindi considerate solo le incertezze dovute all'analisi statistica dei risultati. 

Il set-up per l'esperimento è costituto da un cilindro graduato riempito di glicerina, e  da un cronometro digitale. L'incertezza associata a quest'ultimo strumento è $\pm0.01s$. Abbiamo quindi segnato due tacche sul cilindro graduato, per facilitare la lettura del tempo di caduta, poste ad una distanza di $15.1 cm$, distanza misurata con un metro a nastro di sensibilità $1mm$.

\subsection{Contenuti teorici}
La viscosità è l'attrito dinamico che si oppone al moto di un corpo all'interno di un fluido. Per un moto laminare, la forza di attrito è proporzionale alla velocità e alle caratteristiche del corpo secondo la seguente formula:

$$F_{res} = 6 \pi r v \eta $$

con \textbf{r}  raggio del corpo,\textbf{ v} velocità e $\nu$ indice di viscosità del fluido. 
Per calcolare questo coefficente di viscosità, si può misurare quindi la velocità del corpo nel fluido, che ad un certo punto risulterà costante. La sommatoria delle forze agenti sul corpo è quindi uguale a zero

$$ R = F_{peso} - F_{archimede} - F_{res} = 0 $$
Da questa uguaglianza si ricava:
$$ \eta= 2 g r^2\frac{(\rho_{sfera} - \rho_{liquido})}{9v_{limite}}$$

Per ricavare la velocità limite, si divide la distanza percorsa per il tempo di caduta. 

\section{Raccolta dati}


\subsection{Dati sperimentali}
Di seguito le tabelle che registrano il tempo di caduta delle sfere.
\subsubsection{Sferette da 6mm}
\begin{tabular}{c|c|c|c|c|c|c|c|c|c}
\toprule
 1,68s & 1,68s & 1,65s & 2,21s* & 1,68s & 1,68s & 1,62s & 1,71s & 1,71s & 1,75s \\
 1,78s & 1,71s & 1,71s & 1,65s & 1,59s  & 1,65s  & 1,71s  & 1,68s & 1,68s & 1,65s \\
\midrule
 1,62s & 1,56s & 1,71s & 1,71s & 1,68s & 1,62s & 1,59s & 1,59s & 1,53s & 1,59s \\
 1,46s & 1,62s & 1,59s & 1,56s & 1,62s & 1,56s & 1,62s & 1,43s & 1,53s & 1,59s \\
\midrule
 1,56s & 1,56s & 1,53s & 1,62s & 1,68s & 1,56s & 1,56s & 1,59s & 1,59s & 1,50s\\
 1,53s & 1,62s & 1,50s & 1,56s & 1,59s & 1,62s & 1,43s & 1,62s & 1,43s & 1,40s \\
\midrule
 1,59s & 1,46s & 1,46s & 1,59s & 1,65s & 1,59s & 1,68s & 1,43s & 1,53s & 1,62s \\
 1,46s & 1,50s & 1,56s & 1,59s & 1,50s & 1,59s  & 1,59s & 1,59s & 1,53s & 1,59s \\
\midrule
1,59s & 1,46s & 1,53s & 1,43s & 1,53s & 1,46s & 1,46 & 1,56 & 1,40s & 1,34s \\
1,46s & 1,37s  & 1,50s   & 1,50s  & 1,43s & 1,31s & 1,40s  & 1,43s  & 1,34s & 1,43s \\
\bottomrule
\end{tabular}
\subsubsection{Sferette da 5mm}
\begin{tabular}{c|c|c|c|c|c|c|c|c|c}
\toprule
 2,34s* & 2,43s* & 2,31s & 2,21s & 2,12s & 2,53s* & 2,21s & 2,21s & 2,71s* & 2,71s* \\
 2,12s & 2,34s & 2,18s & 2,34* & 2,12s & 2,06s & 2,12s & 2,59s* & 2,03s & 2,06s \\
\midrule
 2,87s* & 2,21s & 2,21s & 2,12s & 2,03s & 2,09s & 2,12s & 2,46* & 2,06s & 2,25s \\
 2,15s & 2,25s & 2,00s &  &  &  &  &  &  & \\
\bottomrule
\end{tabular}
\subsubsection{Sferette da 4mm}
\begin{tabular}{c|c|c|c|c|c|c|c|c|c}
\toprule
 3,06s & 3,09s & 3,15s & 3,12s & 3,03s & 3,15s & 3,25s & 3,25s & 3,18s & 3,09s \\
 3,09s & 3,09s & 3,21s & 3,12s & 3,06s & 3,06s & 3,15s & 3,46s & 3,25s & 3,12s \\
\bottomrule
\end{tabular}
\subsubsection{Sferette da 3mm}

\begin{tabular}{c|c|c|c|c|c|c|c|c|c}
\toprule
 5,28s & 5,09s & 4,96s & 5,25s & 5,28s & 5,12s & 5,06s & 5,12s & 5,06s & 5,28s \\
 5,03s & 5,06s & 4,87s & 4,90s & 4,96s & 5,06s & 5,06s & 4,95s & 4,93s & 5,18s \\
 5,18s & 5,18s & 5,21s & 4,87 & 5,34s &  &  &  &  & \\
\bottomrule
\end{tabular}

* = per queste misurazioni, la pallina si è avvicinata molto alla parete del tubo, e dunque le misurazioni potrebbero essere falsate.
Per le sferette da 6mm abbiamo scartato il valore 2,21s, connotato dall'asterisco, perchè esso dista circa 6,5 deviazioni standard dal valore medio.

\section{Elaborazione dei dati}
\subsection{Calcolo dei tempi}
Di seguito, gli istogrammi relative alle misure, e la gaussiana associata.
\\
\includegraphics[scale=0.4]{"../grafici/3mm"}
\includegraphics[scale=0.4]{"../grafici/4mm"}
\\
\includegraphics[scale=0.4]{"../grafici/5mm"}
\includegraphics[scale=0.4]{"../grafici/6mm"}

\subsection{Calcolo delle velocità}
Per calcolare la velocità, si divide la distanza ( $ d= 15.1 cm$) per il tempo medio calcolato in precedenza:
$$ v_{limite} = \frac{d}{t_{\mu}} $$
L'incertezza sulle due misure si propaga secondo la seguente formula:
\begin{equation}
\displaystyle \frac{\sigma v}{v} = \sqrt{\displaystyle (\frac{\sigma d}{d})^2 + \displaystyle (\frac{\sigma_t}{t})^2} 
\end{equation}

\begin{center}
\begin{tabular}{c|c|c|c|c}
Sfera & Tempo (s) & $\sigma_t (s) $ & Velocità (m/s) & $\sigma_v (m/s) $\\
\midrule
3mm & 5.09 & 0.13 & 2.96 & 0.08\\
4mm & 3.15 & 0.10 & 4.79 & 0.15\\
5mm & 2.16 & 0.11 & 6.99 & 0.36\\
6mm & 1.56 & 0.10 & 9.68 & 0.62\\

\end{tabular}
\end{center}



\subsection{Conclusioni}
