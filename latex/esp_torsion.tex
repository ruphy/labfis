\documentclass[a4paper,10pt]{report}
 

\usepackage{amsmath}
\usepackage{amsthm}
\usepackage{amssymb}
\usepackage{booktabs}
\usepackage[table]{xcolor}
\usepackage{amssymb}


\renewcommand\chaptername{Esperimento}
 
 
% Title Page
\title{Esperimento}

\begin{document}

\maketitle

\chapter{Pendolo a torsione}

\subsubsection{ strumenti}
micrometro: $ \pm 1 \mu m$
metro: $\pm 1 mm$
sensore di rotazione: $\pm 0.09 °$

\subsubsection{Dati geometrici}
Massa scorrevole: 0.074 kg
Lunghezza sbarra: 0.38 m
Massa anello: 0.46927 kg
Raggio interno: 0.0265 m
Raggio esterno: 0.0355 m

Massa Sbarra: 0.2563 kg
Raggio Sbarra: 0.0045 m
Supporto 0.0035 m

Massa disco: 0.12055 kg
Raggio disco: 0.0475 m
diametro carrucola: 29 mm

\section{ Brevi accenni teorici}
La costante di proporzionalità C \'e detta costante di torsione del filo ed \'e legata ai parametri fisici del filo tramite l'equazione:


 $$ C = G \frac{\pi}{2}\frac{r^4}{l} $$

$$ T = 2\pi \sqrt{\frac{I}{c}}$$


dove r e l sono rispettivamente il raggio e la lunghezza del filo, e G \'e il modulo di rigidit\'a o di scorrimento ed \'e una proprietà specifica del materiale di cui il filo \'e realizzato.

Questo esperimento si suddivide in due parti:
\subsection{Misure dei momenti d'inerzia}
In questa prima parte dell'esperienza si misureranno i momenti d'inerzia dei seguenti corpi, elencati con i rispettivi momenti teorici:
$ 1/2 m R^2 $
$ 1/2 m (R^2_1 + R^2_2) $
$ 1/12 m L^2 + 2 \mu D^2  $
\subsubsection*{Procedimento}
Al fine di calcolare sperimentalmente il momento d'inerzia dei seguenti corpi,
\begin{itemize}
\item disco di raggio e massa
\item anello di raggi e massa
\item una sbarra cilindrica omogenea, con due masse uguali, scorrevoli su di essa, poste equidistanti dall’asse di rotazione, di lunghezza e massa
\end{itemize}



Ne misuriamo la posizione angolare in funzione del tempo, 

I corpi vengono fissati a turno sul perno di una puleggia A posta su un sensore di moto rotatorio, collegato tramite interfaccia al PC. Per mettere in rotazione la puleggia si deve tirare il filo avvolto attorno ad essa, applicando una forza. Si utilizza la forza peso esercitata da masse calibrate sospese all’altro capo del filo. Per fare questo si deve montare sulla base del sensore di moto, lateralmente, una seconda puleggia B, su cui si fa scorrere il filo proveniente dalla puleggia del sensore di moto. All’ estremità del filo si lega il supporto di sospensione delle masse calibrate.
Quando si pongono sul supporto una o più masse il sistema si mette in rotazione. Il sensore di moto rotatorio misura la posizione della puleggia (e dei corpi ad essa solidali) in funzione del tempo, e quindi la velocità angolare  al grafico di  in funzione del tempo si può ricavare l’accelerazione  per derivazione.
Il modulo del momento della forza applicata è calcolabile come:

dove T è la tensione del filo ed r è il braccio della forza, cioè il raggio della puleggia A. La tensione del filo può essere ricavata risolvendo l’equazione del moto della massa ms che cade con accelerazione lineare a:

che permette di ricavare I dalla misura dell’accelerazione  del raggio r e della massa. La misura di I può essere ripetuta variando la massa sospesa, e quindi variando l’accelerazione angolare. Confrontare i risultati ottenuti con il momento di inerzia calcolato a partire dai parametri geometrici del corpo.
Ripetere il procedimento per ciascuno dei corpi a disposizione. L’anello va fissato sopra il disco in modo da misurare il momento di inerzia totale dei due corpi.

\subsection{Pendolo a torsione Misura statica}

\subsection{Pendolo a torsione Misura dinamica}


Periodo filo C:

88.5 / ( 24 max) = 3.875 

\subsection{Filo A}
Diametro: 1,750 mm
Lunghezza: 41,3 cm

50g/ 4,0°
100g/7,0°
200g/ 13,0°

\subsection{Filo B}
Diametro: 1,175 mm
Lunghezza: 43 cm
0 g/ 2,0
50g/ 12,0 °
100g/ 23,0°
200g/47,0°
\subsection{Filo C}
Diametro: 0,880 mm
Lunghezza: 33,5 cm
50 g / 31°
100 g / 61°
200g / 126°
dadati


\section{Raccolta dati}
Abbiamo effettuato una serie di misurazioni 
\section{Analisi dei dati}
\section{Conclusioni}

\end{document} 


