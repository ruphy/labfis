\documentclass[a4paper,10pt]{report}
 
\usepackage[utf8]{inputenc}
\usepackage{amsmath}
\usepackage{amsthm}
\usepackage{amssymb}
\usepackage{booktabs}
\usepackage[table]{xcolor}
\usepackage{amssymb}

\renewcommand\chaptername{Esperimento}
 
\newtheorem{teo}{Teorema}
\newtheorem{defi}{Definizione}
\newtheorem{lemma}{Lemma}
\newtheorem{cor}{Corollario}
\newtheorem{oss}{Osservazione}
\newtheorem{prop}{Proprietà}
% \newtheorem{def}[cap]{×}
 
 
% Title Page
\title{Esperimento}

\begin{document}

\maketitle

\chapter{Viscosità}
\subsection{Da fare}
\begin{itemize}
 \item Grafici con barrette di errore
\end{itemize}

\section{Preambolo}
\subsection{Obiettivo di ricerca}
\subsection{Strumenti di laboratorio}

\section{Misurazioni preliminari}
\subsection{Sferette}
Abbiamo raccolto le seguenti misurazioni per quanto riguarda il diametro delle sferette:
\\

\begin{tabular}{|l|l|l|}
 \toprule
 Sferette tipo 1: & 3mm & 3mm \\
 Sferette tipo 2: & 4mm & 4mm \\
 Sferette tipo 3: & 5mm & 5mm \\
 Sferette tipo 4: & 6mm & 6mm \\
 \bottomrule
\end{tabular}
\\
\\
La sensibilità del calibro è di 0.05mm.\\
Peso delle sferette:
\\

\begin{tabular}{lll}
 Sferette tipo 1:
 & 0.1102g & 0.1102g \\
 & 0.1101g & \\
 \midrule
 Sferette tipo 2: 
 & 0.2612g & 0.2613g \\
 & 0.2613g & 0.2612g \\
 \midrule
 Sferette tipo 3: 
 & 0,5098g & 0,5100g \\
 & 0,5094g & 0,5100g \\
 & 0,5096g & 0,5098g \\
 \midrule
 Sferette tipo 4:
  & (3,521g/4) & 0.8806g \\
  & 0.8802g & 0.8807g \\
\end{tabular}
\\
\\
La sensibilità della bilancia è di $10^{-6}$ kg

In particolare, per una sferetta singola da 5mm, abbiamo ottenuto le seguenti misurazioni\\
\begin{tabular}{ccccccc}
0,5096g & 0,5098g & 0.5099g & 0,5095g & 0,5095g & 0,5096g & 0,5097g
\end{tabular}
\\
Per quanto riguarda il cilindro di glicerina, abbiamo segnato due tacche a una distanza di 15,1mm l'una dall'altra.
Per misurare questa distanza abbiamo utilizzato un metro a nastro con una sensibilità di 1mm.

\subsection{Dati sperimentali}
Di seguito le tabelle che registrano il tempo impiegato per attraversare il fluido.
\subsubsection{Sferette da 6mm}
\begin{tabular}{|c|c|c|c|c|c|c|c|c|c|}
\toprule
 1,68s & 1,68s & 1,65s & 2,21s* & 1,68s & 1,68s & 1,62s & 1,71s & 1,71s & 1,75s \\
 1,78s & 1,71s & 1,71s & 1,65s & 1,59s  & 1,65s  & 1,71s  & 1,68s & 1,68s & 1,65s \\
\midrule
 1,62s & 1,56s & 1,71s & 1,71s & 1,68s & 1,62s & 1,59s & 1,59s & 1,53s & 1,59s \\
 1,46s & 1,62s & 1,59s & 1,56s & 1,62s & 1,56s & 1,62s & 1,43s & 1,53s & 1,59s \\
\midrule
 1,56s & 1,56s & 1,53s & 1,62s & 1,68s & 1,56s & 1,56s & 1,59s & 1,59s & 1,50s\\
 1,53s & 1,62s & 1,50s & 1,56s & 1,59s & 1,62s & 1,43s & 1,62s & 1,43s & 1,40s \\
\midrule
 1,59s & 1,46s & 1,46s & 1,59s & 1,65s & 1,59s & 1,68s & 1,43s & 1,53s & 1,62s \\
 1,46s & 1,50s & 1,56s & 1,59s & 1,50s & 1,59s  & 1,59s & 1,59s & 1,53s & 1,59s \\
\midrule
1,59s & 1,46s & 1,53s & 1,43s & 1,53s & 1,46s & 1,46 & 1,56 & 1,40s & 1,34s \\
1,46s & 1,37s  & 1,50s   & 1,50s  & 1,43s & 1,31s & 1,40s  & 1,43s  & 1,34s & 1,43s \\
\bottomrule
\end{tabular}
\subsubsection{Sferette da 5mm}
\begin{tabular}{|c|c|c|c|c|c|c|c|c|c|}
\toprule
 2,34s* & 2,43s* & 2,31s & 2,21s & 2,12s & 2,53s* & 2,21s & 2,21s & 2,71s* & 2,71s* \\
 2,12s & 2,34s & 2,18s & 2,34* & 2,12s & 2,06s & 2,12s & 2,59s* & 2,03s & 2,06s \\
\midrule
 2,87s* & 2,21s & 2,21s & 2,12s & 2,03s & 2,09s & 2,12s & 2,46* & 2,06s & 2,25s \\
 2,15s & 2,25s & 2,00s &  &  &  &  &  &  & \\
\bottomrule
\end{tabular}
\subsubsection{Sferette da 4mm}
\begin{tabular}{|c|c|c|c|c|c|c|c|c|c|}
\toprule
 3,06s & 3,09s & 3,15s & 3,12s & 3,03s & 3,15s & 3,25s & 3,25s & 3,18s & 3,09s \\
 3,09s & 3,09s & 3,21s & 3,12s & 3,06s & 3,06s & 3,15s & 3,46s & 3,25s & 3,12s \\
\bottomrule
\end{tabular}
\subsubsection{Sferette da 3mm}
\begin{tabular}{|c|c|c|c|c|c|c|c|c|c|}
\toprule
 5,28s & 5,09s & 4,96s & 5,25s & 5,28s & 5,12s & 5,06s & 5,12s & 5,06s & 5,28s \\
 5,03s & 5,06s & 4,87s & 4,90s & 4,96s & 5,06s & 5,06s & 4,95s & 4,93s & 5,18s \\
 5,18s & 5,18s & 5,21s & 4,87 & 5,34s &  &  &  &  & \\
\bottomrule
\end{tabular}
\\
\\
* = per queste misurazioni, la pallina si è avvicinata molto alla parete del tubo, e dunque le misurazioni potrebbero essere falsate.
TODO: rimuovi i vecchi valori

\section{Elaborazione dei dati}
\subsection{Calcolo dei tempi}
Per le sferette da 6mm abbiamo scartato il valore 2,21s, connotato dall'asterisco, in quanto evidente errore sperimentale. A conferma di ciò, si trova che esso esso dista circa 6,5 deviazioni standard dal valore medio.

[tabelle statistiche]
- Gaussiane+istogrammi
- Plotting dei dati
- Tabelle di conti (frequenze assolute, relative, ...)

\subsubsection{Note operative}
Abbiamo eliminato alcuni dati

\subsection{Estrapolazione}
- Grafici velocità (lin, quad e log)
- Calcolo di astar, bstar

\subsection{Sezione teorica}
Abbiamo usato le seguenti formule statistiche per....

\subsection{Conclusioni}

%%% ----------------------------------

\chapter{Pendolo a torsione}

\subsubsection{ strumenti}
micrometro: $ \pm 1 \mu m$
metro: $\pm 1 mm$
sensore di rotazione: $\pm 0.09$ gradi

\subsubsection{Dati geometrici}
\begin{tabular}{ll}
Masse puntiformi: & 0.074 kg\\
Lunghezza sbarra: & 0.38 m\\
Massa Sbarra: & 0.2563 kg\\
Raggio Sbarra: & 0.0045 m\\

\midrule

Massa anello: & 0.46927 kg\\
Raggio interno: & 0.0265 m\\
Raggio esterno: & 0.0355 m\\

\midrule

Supporto & 0.0035 m\\

\midrule

Massa disco: & 0.12055 kg\\
Raggio disco: & 0.0475 m\\
Diametro carrucola: & 29 mm\\

\end{tabular}

\section{Brevi accenni teorici}
La costante di proporzionalità $c$ \'e detta costante di torsione del filo ed \'e legata ai parametrié fisici del filo tramite l'equazione:


$$ c = G \frac{\pi}{2}\frac{r^4}{l} $$

$$ T = 2\pi \sqrt{\frac{I}{c}}$$


dove $r$ e $l$ sono rispettivamente il raggio e la lunghezza del filo, $I$ è il momento d'inerzia, $T$ il periodo, e $G$ è il modulo di rigidità o di scorrimento ed è una proprietà specifica del materiale di cui il filo è realizzato.

Dunque
$$ c = 4\pi^2\frac{I}{T^2} $$

Questo esperimento si suddivide in due parti:
\subsection{Misure dei momenti d'inerzia}
In questa prima parte dell'esperienza si misureranno i momenti d'inerzia dei seguenti corpi, elencati con i rispettivi momenti teorici:
$ 1/2 m R^2 $
$ 1/2 m (R^2_1 + R^2_2) $
$ 1/12 m L^2 + 2 \mu D^2  $
\subsubsection*{Procedimento}
Al fine di calcolare sperimentalmente il momento d'inerzia dei seguenti corpi,
\begin{itemize}
\item disco di raggio e massa
\item anello di raggi e massa
\item una sbarra cilindrica omogenea, con due masse uguali, scorrevoli su di essa, poste equidistanti dall’asse di rotazione, di lunghezza e massa
\end{itemize}



Ne misuriamo la posizione angolare in funzione del tempo, 

I corpi vengono fissati a turno sul perno di una puleggia A posta su un sensore di moto rotatorio, collegato tramite interfaccia al PC. Per mettere in rotazione la puleggia si deve tirare il filo avvolto attorno ad essa, applicando una forza. Si utilizza la forza peso esercitata da masse calibrate sospese all’altro capo del filo. Per fare questo si deve montare sulla base del sensore di moto, lateralmente, una seconda puleggia B, su cui si fa scorrere il filo proveniente dalla puleggia del sensore di moto. All'estremità del filo si lega il supporto di sospensione delle masse calibrate.
Quando si pongono sul supporto una o più masse il sistema si mette in rotazione. Il sensore di moto rotatorio misura la posizione della puleggia (e dei corpi ad essa solidali) in funzione del tempo, e quindi la velocità angolare  al grafico di [] in funzione del tempo si può ricavare l’accelerazione  per derivazione.
Il modulo del momento della forza applicata è calcolabile come:

dove T è la tensione del filo ed r è il braccio della forza, cioè il raggio della puleggia A. La tensione del filo può essere ricavata risolvendo l’equazione del moto della massa ms che cade con accelerazione lineare a:

che permette di ricavare I dalla misura dell’accelerazione  del raggio r e della massa. La misura di I può essere ripetuta variando la massa sospesa, e quindi variando l'accelerazione angolare. Confrontare i risultati ottenuti con il momento di inerzia calcolato a partire dai parametri geometrici del corpo.
Ripetere il procedimento per ciascuno dei corpi a disposizione. L’anello va fissato sopra il disco in modo da misurare il momento di inerzia totale dei due corpi.

\subsection{Pendolo a torsione Misura statica}

\subsection{Pendolo a torsione Misura dinamica}

\subsection{Filo A}
\begin{tabular}{ll}
Diametro: & 1,750 mm\\
Lunghezza: & 41,3 cm\\
\end{tabular}
\\ \\
Le misurazioni sono state le seguenti:
\begin{center}
\begin{tabular}{|l|l|}
\toprule
Peso (g) & Rotazione (gradi)\\
\midrule
50g & 4.0 \\
100g & 7.0 \\
200g & 13.0 \\
\bottomrule
\end{tabular}
\end{center}

\begin{verbatim}
= Period (in s) =
Mean = 0.644

String preformatted for GNU Octave:
values = [ 0.65 0.65 0.65 0.625 0.65 0.675 0.65 0.625
0.65 0.65 0.65 0.625 0.65 0.65 0.625 0.65
0.65 0.65 0.625 0.675 0.625 0.65 0.625 0.65 0.65 0.65
0.65 0.625 0.65 0.675 0.625 0.65 0.65 0.625
0.625 0.675 0.65 0.65 0.625 0.65 0.625 0.65 0.625 0.65
0.65 0.625 0.65 0.65 0.65 0.625]
\end{verbatim}

\subsection{Filo B}
\begin{tabular}{ll}
Diametro:  & 1,175 mm \\
Lunghezza: & 43 cm \\

\end{tabular}
\\ \\
Le misurazioni sono state le seguenti:
\begin{center}
\begin{tabular}{|l|l|}
\toprule
Peso (g) & Rotazione (gradi)\\
\midrule
50g & 12.0 \\
100g & 23.0 \\
200g & 47.0 \\
\bottomrule
\end{tabular}
\end{center}
Periodo di oscillazione (s):
\\
\begin{center}
\begin{tabular}{llllllll}
2.4   & 2.4   & 2.425 & 2.4  & 2.4   & 2.425 & 2.4  & 2.4 \\
2.425 & 2.425 & 2.425 & 2.45 & 2.45  & 2.5   & 2.5  & 2.45 \\
\end{tabular}
\end{center}
Media = 2.43

[Grafico Octave]

\subsection{Filo C}
\begin{tabular}{ll}
Diametro: & 0,880 mm \\
Lunghezza: & 33,5 cm \\
\end{tabular}
\\ \\Le misurazioni sono state le seguenti:
\begin{center}
\begin{tabular}{|l|l|}
\toprule
Peso (g) & Rotazione (gradi)\\
\midrule
50g & 31.0 \\
100g & 61.0 \\
200g & 126.0 \\
\bottomrule
\end{tabular}
\end{center}

\begin{verbatim}
= Period (in s) =
Mean = 3.844

String preformatted for GNU Octave:
values = [ 3.875 3.9 3.9 3.875 3.9 3.875 3.875
3.875 3.875 3.85 3.875 3.85 3.85 3.85 3.825
3.825 3.825 3.8 3.8 3.75 3.775 3.75]
\end{verbatim}


\section{Raccolta dati}
Abbiamo effettuato una serie di misurazioni 
\section{Analisi dei dati}
\section{Conclusioni}
\end{document} 


